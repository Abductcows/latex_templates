\section{Υλοποίηση του \lt Gauss με Σειριακό Προγραμματισμό}

%%%%%%%%%%%%%%%%%%%%%%%%%%%%%%%%%%%%%
\subsection{Aλγόριθμοι στο \lt{{\sc{LaTex}}}}    

\begin{algorithm}[H]
 \footnotesize {
\begin{algor} : {Πολλαπλασιασμός του \lt{Karatsuba}}\\
{\bf Είσοδος.} $a,b$ ακέραιοι\\
{\bf Έξοδος.} $a\cdot b$
\end{algor}
{\lt{
 \SetAlgoLined
\nl def karatsuba(a, b)\\
\nl\ \If {$a<100$ or $b<100$}{
     return $a\cdot b$
     }    
\nl\ $m=\max(\log_{10}(a),\log_{10}(b))$\\
\nl\ $m_2=floor(m/2)$\\
\nl\  $high(a)$ = take the first $m_2$ decimal digits of $a$\\
\nl\  $low(a)$ = take the last $m_2$ decimal digits of $a$\\
\nl\  $high(b)$ = take the first $m_2$ decimal digits of $b$\\
\nl\  ...\\
\nl\  ...\\
\nl\  ....\\
\nl  print $(z_2\cdot 10^{2m_2}+(z1-z2-z0)\cdot 10^{m_2}+z_0$}}
              }
\end{algorithm}

\newpage

{\footnotesize{
\noindent
\begin{algor} : {Αλγόριθμος Απαρίθμησης (\lt{KFP enumeration algorithm})}\label{al-enum}\\
{\bf Είσοδος.}  Μια διατεταγμένη βάση $\mathcal{B}=\{{\bf b}_1,\dots, {\bf{b}}_n\}\subset\mathbb{Z}^{m}$ 
του πλέγματος $\mathcal{L}(\mathcal{B})$ και ένα θετικό αριθμό $R.$ \\
{\bf Έξοδος.} Όλα τα διανύσματα ${\bf x}\in \L$ με $||{
\bf x}||\leq R.$
\end{algor}
\lt{\ \\
	\texttt{01.}   {\texttt{Compute $\{\mu_{ij}\}$ and $B_i=||{\bf b}_i^{*}||^2$}}\\
    \texttt{02.}   {\texttt{${\bf x}=(x_i)\leftarrow {\bf 0}_n$,${\bf c}=(c_i)\leftarrow {\bf 0}_n$, ${\bf \ell}=(\ell_i)\leftarrow {\bf 0}_n$, $sumli\leftarrow 0, S=\emptyset, i\leftarrow 1$}\\
    \texttt{03.}  \texttt{{\bf While}} $i\leq n$\\
	\texttt{04.}   \hspace*{0.4cm} $c_{i}\leftarrow -\sum_{j=i+1}^n x_j\mu_{ji}$\\
	\texttt{05.}  ...\\ \\ \\
	\texttt{19.}	{\texttt{return S}}\\
}}}
